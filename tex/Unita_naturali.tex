% !TEX root = main.tex

\chapter{Unità Naturali}

Portare avanti nelle epressioni matematiche tutte le costanti fondamentali può essere pesante. Pertanto si sceglie di non considerarle quando considerazioni dimensionali permettono di ricostruire quali costanti vanno inserite a posteriori. E' questo il caso della meccanica quantistica e relativistica, dove risulta conveniente porre:

\begin{equation}
c=\hbar=1
\end{equation}

Visto che poi è possibile ricostruire quali sono le potenze mancanti di queste costanti all'interno delle formule.

Nella tabella 2.1 sono riassunti i valori di tali costanti nelle più comuni unità di misura. Conoscendo tali valori è possibile effettuare i calcoli a posteriori.
Le unità di misura ottenute in questo modo, unitamente alla definizione della costante di struttura fine $\alpha_{em}=\frac{e^2}{4\pi\epsilon_0\hbar c}\sim\frac{1}{137}$ che, poste $c=\hbar=1$ diventa $\alpha_{em}=\frac{e^2}{4\pi\epsilon_0}$, costituiscono il sistema di unità naturali.

\begin{table}
\center
\begin{tabular}{|l|l|l|}
\hline
                                                            & Dimensioni                                             & Misura                           \\ \hline
\multicolumn{1}{|c|}{\multirow{3}{*}{$\hbar$}} & \multirow{3}{*}{energia$\cdot$ tempo}     & $1,035 \cdot 10^{-34} J\cdot s$  \\ \cline{3-3} 
\multicolumn{1}{|c|}{}                                      &                                                        & $6,5 \cdot 10^{-16} eV\cdot s$   \\ \cline{3-3} 
\multicolumn{1}{|c|}{}                                      &                                                        & $6,5 \cdot 10^{-13} MeV\cdot ns$ \\ \hline
\multirow{4}{*}{c}                                          & \multirow{4}{*}{posizione/tempo}                       & $3,0\cdot10^8 m/s$               \\ \cline{3-3} 
                                                            &                                                        & $300 km/s$                       \\ \cline{3-3} 
                                                            &                                                        & $\mathbf{30 cm/ns}$              \\ \cline{3-3} 
                                                            &                                                        & $3,0\cdot10^{14} fm/ns$          \\ \hline
\multirow{2}{*}{$\hbar$ c}                     & \multirow{2}{*}{energia$\cdot $posizione} & $3,1\cdot 10^{-26} J\cdot m$     \\ \cline{3-3} 
                                                            &                                                        & $\mathbf{200 MeV \cdot fm}$            \\ \hline
\end{tabular}
\caption{Valori delle costanti $\hbar$ e $c$ nelle più comuni unità di misura}
\end{table}

Il metodo più semplice per convertire le grandezze da unità naturali a sistema internazionale, è porre:

\begin{equation}
Q_{SI}=Q_{UN}\cdot(\hbar c)^m\cdot c^n
\end{equation}

Dove $Q_{UN}$ è una qualsiasi grandezza espressa nel sistema di unità naturali, $Q_{SI}$ è la stessa grandezza espressa nel sistema di unità internazionale e $m$ ed $n$ sono le specifiche potenze che bisogna porre ai termini $\hbar c$ e $c$ affinché le dimensioni della grandezza considerata siano coerenti in entrambi i sistemi di unità di misura.\\
Si consiglia di utilizzare i valori delle costanti espresse in grassetto nella tabella 2.1.

\section{Cinematica e Dinamica Relativistica}

Il sistema di unità naturali è specialmente utile perché semplifica le equazioni della relatività ristretta. Nella tabella 2.2 è presente una comparazione di tale equazioni nel sistema di unità internazionale e in quello di unità naturali

\begin{table}
\centering
\begin{tabular}{|l|l|}\hline
SI & UN \\ \hline
$\beta=\frac{v}{c}$   & $\beta=v$   \\ \hline
$E=\sqrt{p^2c^2+m^2c^4}$   &   $E=\sqrt{p^2+m^2}$ \\ \hline
$p=m\beta\gamma c$   &  $p=m\beta\gamma$  \\ \hline
$T=E-mc^2$   &  $T=E-m$  \\ \hline 
   &  $\beta=\frac{p}{E}$  \\ \hline
   &  \\ \hline
\end{tabular}
\label{equazioni}
\caption{Confronto delle equazioni di cinematica relativistica nei due sistemi di unità di misura}
\end{table}

\subsection{Esempi ed Applicazioni}

\emph{Un elettrone ha quantità di moto $p_{UN}=1 MeV/c$. Quale è la sua quantità di moto espressa nel sistema internazionale?}

Ricordando la conversione $1 eV=1,6\cdot10^{-19} J$ e il valore della velocità della luce $c=3\cdot 10^8 m/s$, otteniamo:

\begin{equation}
p_{SI}=p_{UN}\cdot\frac{1,6\cdot10^{-19}\cdot10^6}{3\cdot10^8}=5\cdot10^{-22}kg\cdot m/s
\end{equation}

\emph{Qual è l'energia elettrostatica di un elettrone a distanza $d=0,5$\AA da un nucleo di carbonio, trascurando lo schermaggio prodotto dagli altri elettroni?}

Ricordando l'espressione per l'energia potenziale elettrostatica per un sistema di due cariche puntiformi abbiamo

\begin{equation}
U=\frac{Ze^2}{4\pi\epsilon_0d}=\frac{Z\alpha_{UN}}{d}
\end{equation}

Dove $\alpha_{UN}=\frac{e^2}{4\pi\epsilon_0}\sim\frac{1}{137}$ è detta costante di struttura fine ed è adimensionale: non dipende dunque dal sistema di unità di misura scelto. Facendo un'analisi dimensionale ora decidiamo quali potenze di $\hbar c$ e $c$ vanno aggiunte.

A sinistra abbiamo un'energia, a destra l'inverso di una lunghezza.

\begin{equation}
E=\frac{([\hbar c])^m \cdot [c]^n}{L}=(E\cdot L)^m\left(\frac{L}{T}\right)^n\cdot L^{-1}
\end{equation}

Da cui si ricava che necessariamente $m=1$ e $n=0$. Ovvero:

\begin{equation}
U=\frac{Z\alpha\hbar c}{d}=\frac{6\cdot\frac{1}{137}\cdot200MeV\cdot fm}{5\cdot 10^4 fm}=175eV
\end{equation}

Dove ci siamo ricordati che 1\AA$=10^{-10}m$ e $1 fm=10^{-15}m$.

Esercizi ulteriori:

\begin{itemize}
\item \emph{Qual è l'energia cinetica di un nucleo di Elio con impulso $p=50MeV/c$?}
\item \emph{Qual è il raggio dell'orbita di un protone con energia cinetica $T=10MeV$ immerso in un campo magnetico di modulo $B=0,5 T$?}
\item \emph{Quali sono la $\beta$ e $\beta\gamma$ di un elettrone accelerato da $1 MV$?}
\end{itemize}

