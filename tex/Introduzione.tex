% !TEX root = main.tex

\chapter{Introduzione ed esempi di applicazioni}

Le applicazioni della fisica delle particelle elementari si basano sull'introduzione di radiazione o materiale radiattivo nell'oggetto in studio. In particolare, nel caso medico, l'oggetto in esame è il paziente e la radiazione può uscire da esso (diagnostica), o interagire al suo interno per fini terapeutici (radioterapia).
Verranno considerate radiazioni di tipo $\beta-$ e $\alpha$, che comportano un rilascio di energia locale, radiazioni di tipo $\beta+$ in cui il positrone creato riemette energia sottoforma di fotoni, e radiazione puramente elettromagnetica.\\
Per un'introduzione approfondita alla fisica medica e alla fisica nucleare consultare nella bibliografia le fonti \cite{ENMP} \cite{NMP}, \cite{CLA}, \cite{Corvisiero3}, \cite{Nutshell}.\\

Si riporta qui una vetrina di possibili applicazioni della fisica della radiazione che saranno utilizzate nel seguito per esemplificare gli argomenti di fisica che si andranno a studiare.

\section{Diagnostica}

La diagnostica può essere \emph{morfologica} se dà informazioni sulla morfologia del corpo (densità, presenza di masse anomale). Esempi sono la CT e la radiografia 2D, che usano i raggi X.\\
Oppure può essere \emph{funzionale} se dà anche informazioni di tipo metabolico. Esempi sono PET e SPECT, che usano decadimenti $\beta+$ che portano a produzione di coppie di fotoni.
La diagnostica funzionale si basa sulla somministrazione di radiofarmaci, ovvero sostanze che vengono assorbite dall'organo di interesse per \emph{metabolismo} o \emph{recettorialità}. Si parla di assorbimento tramite metabolismo quando vengono sfruttate proprietà fisiche dell'organo in esame, come la presenza di membrane che possono essere superate solo da sostanze di dimensioni ridotte. L'assorbimento tramite recettorialità invece si basa sulle proprietà chimiche dei recettori presenti nelle membrane dei tessuti di interesse. Radiofarmaci specifici, infatti, possono simulare le caratteristiche chimiche di uno specifico ligando, reagendo al suo posto con i recettori nelle membrane cellulari dei tessuti dell'organo in esame, mantenendo così la sostanza radioattiva in vicinanza delle cellule che si vogliono esaminare.

 Esempi pratici:
\begin{itemize}
\item Single Photon Emission Computed Tomography (SPECT). Fa utilizzo di $^{131}\text{I}$ o $^{99m}\text{Tc}$. Ad esempio il Tecnezio viene rapidamente smaltito nei reni. Una volta lì esso emette raggi $\gamma$ di $140 keV$ di energia. Con una tecnica chiamata tomografia è possibile dedurne la posizione e ricostruire il metabolismo dei reni.
\item Positron Emission Tomography (PET). Tipicamente fa uso di $^{18}\text{F}$ legato a molecole di zucchero che vengono assorbite dal cervello e da tumori sia primari che metastatici. Permette anche di determinare lo stadio del tumore, e nel particolare se durante la terapia il tumore sta recedendo o meno. Il funzionamento della PET si basa sul fatto che l'isotopo del Fluoro $^{18}\text{F}$ decade $\beta+$. I positroni che vengono emessi da tale processo, interagendo con gli elettroni nelle vicinanze, possono o annichilire in volo o formare positronio. In entrambi i casi vengono emesse coppie di fotoni $\gamma$ che possono essere sfruttate per ricostruire con tecniche tomografiche la posizione dell'isotopo del Fluoro e di conseguenza dei tessuti che lo hanno assorbito.
\item Chirurgia Radio-Guidata. Evidenzia i tumori per aiutarsi nella loro rimozione. Si usa spesso per i linfonodi. L'ultima frontiera è l'utilizzo di nuclidi che decadono $\beta-$.
\end{itemize}

Approfondimenti sulla PET nella bibliografia: \cite{PET1} \cite{PET2}.\\

Approfondimento sulla CT: \cite{CT}\\

Approfondimenti sulla chirurgia radioguidata: \cite{Radiosurgery} \cite{Intraoperative_probes}.

\section{Radioterapia}

La Radioterapia si basa sulla distruzione del meccanismo riproduttivo del tumore, ovvero del DNA. Esso va rotto in almeno due punti o si potrà riprodurre. La radioterapia convenzionale, ottenuta irraggiando il paziente con fotoni o elettroni, non ha energia sufficiente a rompere direttamente il DNA. Piuttosto crea radicali liberi rompendo molecole d'acqua che poi intossicano la cellula tumorale. \`E una sorta di chemioterapia localizzata. Studieremo la differenza nel rilascio di energia tra fotoni e adroni e perché è più conveniente usare questi ultimi.\\

Altri due tipi di radioterapia sono la terapia radiometabolica che si basa sulla distruzione direttamente dall'interno di un tumore sfruttando un radiofarmaco metabolico e la brachiterapia, che consiste nell'applicazione locale sulla pelle di pomate formate da nuclei radioattivi.\\

Approfondimenti sull'adroterapia: \cite{Rossi} \cite{Braccini} \cite{ADRO} \cite{Boron} \cite{Acceleratori_adronterapia} \cite{LINAC}\\

Approfondimento sulla brachioterapia: \cite{Brachioterapia}

\section{Teragnostica}

Con il termine 'teragnostica' si indicano tutte quelle tecniche che tentano di attuare una terapia sul paziente estraendo allo stesso tempo informazioni sulla morfologia e la posizione dei tumori da attaccare. 
Si fa con elementi che emettono contemporaneamente beta- e gamma. Questi ad esempio sono $^{177}\text{Lu}$ e $^{90}\text{Y}$ con $^{68}\text{Ga}$.

\section{Beni Culturali - Ion Beam Analysis (IBA)}

Tecniche analoghe possono essere applicate ai beni culturali. Analizzando l'emissione di radioattività dell'oggetto in esame ad esempio si possono dedurre le seguenti caratteristiche: 

\begin{itemize}
\item Datazione
\item Composizione superficiale
\item Ricerca di contraffatti
\item Corrosione
\item Provenienza
\end{itemize}

Le tecniche di analisi fanno parte della famiglia detta Ion Beam Analysis (IBA), il cui principio base consiste nel bombardare l'oggetto con ioni (generalmente protoni o particelle alfa) e studiare le conseguenze. Questa famiglia contiene al suo interno le seguenti tecniche:
\begin{itemize}
\item Proton Induced X-rays Emission (PIXE), Bombardamento non invasivo di protoni che provocano l'emissione di uno spettro nella regione dei raggi X.
\item Rutherford Backscattering Spectroscopy (RBS), Utile per campioni sottili formati da elementi pesanti. Si analizza la distribuzione dell'angolo con cui le particelle inviate scatterano all'indietro.
\item Elastic Recoil Detection analysis (ERD), Simile al precedente ma funziona per bersagli formati da elementi leggeri.
\item Proton Induced Gamma-rays Emission (PIGE), Come la PIXE ma funziona per elementi leggeri, e si basa sull'emissione di uno spettro nella regione dei raggi gamma.
\end{itemize}

Un'altra tecnica di analisi dei beni culturali che però non si basa su bombardamento di ioni è la X-Ray Fluorescence (XRF), che consiste nel bombardamento del campione con raggi X e nell'analisi della radiazione riemessa per fluorescenza.\\

Approfondimento sui beni culturali: \cite{Cultural}


\section{Datazione col $^{14}\text{C}$}

In natura il Carbonio esiste in tre isotopi, di cui due stabili $^{12}\text{C}$, $^{13}\text{C}$ ed uno radioattivo, $^{14}\text{C}$ con vita media $\tau=8267y$.
La concentrazione media di $^{14}\text{C}$ nell'atmosfera terrestre tende a rimanere costante grazie al continuo flusso di raggi cosmici che vi incide. Infatti  componenti secondari dei raggi cosmici sono i neutroni, che quando si scontrano con i due isotopi stabili del carbonio portano alla produzione di $^{14}\text{C}$.
Nell'atmosfera dunque il rapporto tra numero di isotopi radoattivi e stabili è costante, intorno a $10^{-12}$. \\
Finché un organismo è vivo, esso attua processi di respirazione, fotosintesi o si nutre di altri esseri viventi. In tal modo scambia carbonio con l'atmosfera, mantenendo costante il rapporto tra isotopi del carbonio radioattivi e stabili nel suo corpo. Dopo la sua morte, tali processi cessano. Il carbonio radioattivo decade senza essere riacquistato dall'atmosfera, e dunque il rapporto tra isotopi radioattivi e stabili comincia a scendere in maniera prevedibile.\\
Studiando un organismo deceduto, dunque, si può dedurre in quale periodo storico esso viveva.

\section{Neutroni}

Dato un certo elemento la sua probabilità di interazione con i raggi X è inversamente proporzionale alla sua probabilità di interazione con un neutrone. Allora i neutroni servono laddove i raggi X non riescono a ricavare informazioni sufficienti.
Ad esempio i neutroni vengono assorbiti subito dall'acqua, rendendoli utili per tomografie in cui si voglia evidenziare la presenza di acqua (motori). \\
Approfondimenti sui neutroni: \cite{Corvisiero} \cite{Assay}

